% ----------------------------------------------------------------------
% Página do resumo em Inglês:
\selectlanguage{english}
\vspace*{2cm}
\begin{center}
\Large \bf Abstract
\end{center}
\vspace*{1cm} \setlength{\baselineskip}{0.6cm}

Self-service machines have become an integral component of modern daily life and play a central role in the delivery of various services. They enable operations to be carried out efficiently, autonomously, and accessibly, responding to the demand for solutions that save time and increase efficiency. However, their growing use also brings challenges in the field of security, requiring mechanisms capable of observing, interpreting, and reacting to risk situations in real time.

Within the scope of an internship, a security system for self-service machines was developed with the objective of reinforcing equipment protection and ensuring that operations are carried out in a safe and reliable manner, while also contributing to user security. The solution relies on computer vision techniques and uses the machine’s built-in camera to continuously analyze the surrounding environment and the interactions taking place within it, identifying visual signs of risk and providing information that supports timely and well-informed decisions.

Ease of integration was adopted as a guiding principle during development, allowing the system to be deployed without modifying the machine’s main application and preserving all original functionalities. This approach reduces operational impact, facilitates adoption across different contexts, and lowers implementation costs.

The system was designed with a flexible and evolutive architecture, enabling the efficient integration of new computer vision models and technologies in an agile manner, keeping pace with advances in the field and adapting to future requirements.

The results obtained demonstrate that the proposed solution is viable for deployment in self-service machines, being able to operate continuously and to signal critical events with latency compatible with the operational context. Validation performed using data representative of different usage scenarios confirms consistent performance, achieving a recall value of approximately 0.77, which highlights the system’s ability to effectively identify risk situations without interfering with the normal operation of the machine.
%Resumo até \textbf{300} palavras. 

\vfill

\begin{flushleft}
\textbf{Keywords:}  
Self-service machines, Computer vision, Real-time detection, Behavioral analysis, Continuous monitoring
\end{flushleft}



% ----------------------------------------------------------------------
