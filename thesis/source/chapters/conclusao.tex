% Conclusao

\chapter{Conclusão}


\section{Síntese Final}


O objetivo central desta dissertação foi o desenvolvimento de um sistema de segurança baseado em visão computacional, direcionado para o reforço da proteção em máquinas de autoatendimento. Este objetivo foi concretizado com a conceção de um sistema que acoplou ao produto HEFESTO como camada adicional de segurança, desenvolvido no âmbito de um estágio curricular e capaz de responder a desafios reais associados à deteção automática de ameaças físicas particularmente relevantes em contextos bancários.

A partir da análise do estado da arte e das vulnerabilidades específicas destas máquinas, identificaram-se as limitações das abordagens tradicionais de segurança. Optou-se pela utilização de redes neuronais convolucionais, em particular os modelos YOLO para deteção em tempo real, dada a sua adequação ao problema e a compatibilidade com as restrições de hardware. Evitou-se, por outro lado, o uso de arquiteturas mais pesadas, como os transformadores, privilegiando soluções leves e eficientes, mais ajustadas à infraestrutura disponível.


O sistema desenvolvido agregou deteção de armas, reconhecimento de comportamentos violentos e análise emocional sobre \textit{frames} em tempo real, suportado por uma arquitetura modular \textit{plug-and-play}. Os modelos de inferência pré-treinados foram implementados como serviços externos expostos por uma API REST, o que permitiu ativar, substituir ou atualizar modelos sem alterar a lógica principal do sistema. A integração com as aplicações já existentes nas máquinas de autoatendimento é direta e não exige alterações estruturais: basta que a aplicação capture os \textit{frames} da câmara e os envie, via HTTP, para o sistema de segurança. As decisões são configuráveis, combinando outputs por categoria através de regras de \textit{consensus} ou \textit{scoring}, com limiares e pesos ajustáveis. Além disso, a execução local e a política de não armazenamento garantem a operação independente do sistema, assegurando baixo acoplamento e respeito com os princípios de privacidade.

  
A validação do sistema foi realizada com recurso ao \textit{dataset} criado no âmbito deste trabalho, que combinou dados abertos, encenados e sintéticos, permitindo concluir que o sistema está apto a operar em condições próximas das reais de utilização. Complementarmente, os testes de desempenho mostraram que a execução em CPU permite assegurar funcionamento contínuo do sistema, embora com diferentes níveis de desempenho consoante a capacidade do \textit{hardware} disponível, tendo-se verificado que o envio de \textit{frames} em lote contribuiu para aumentar a eficiência do processamento por \textit{frame}. Entre os pressupostos e metas definidos no início do trabalho, alguns foram confirmados de forma explícita: o sistema opera localmente de forma estável, sem dependência de rede, e demonstrou ser capaz de manter uma taxa de processamento próxima de 1 FPS e tempos médios de processamento por frame abaixo de 1 segundo em ambiente de desenvolvimento, assegurando também valores de \textit{recall} superiores aos 0.7 definidos.
Já no contexto real da máquina de autoatendimento HEFESTO, verificaram-se limitações de desempenho, com \textit{throughput} inferior a 1 \acrshort{fps} e latências superiores a 1 segundo por frame. Apesar de não atingir o valor inicialmente estabelecido, o sistema manteve-se funcional e estável, validando a arquitetura no ambiente operacional, ainda que com compromissos de desempenho inerentes ao \textit{hardware} limitado disponível.


Para além da análise de desempenho, procedeu-se ao ajuste de parâmetros como pesos e limiares, que otimizou a sensibilidade a eventos críticos sem introduzir um número excessivo de alertas indevidos. Verificou-se ainda que a regra de decisão por \textit{scoring} permite incorporar melhor o conhecimento sobre o desempenho relativo de cada modelo, enquanto a opção por \textit{consensus} mantém utilidade em cenários de menor informação.

Adicionalmente, funcionalidades como a métrica de proximidade facial fornecem informação adicional que não interfere na decisão de alarme, mas emite alertas contextuais sobre a presença de pessoas próximas do utilizador. Esta capacidade contribuiu para enriquecer a perceção da situação e poder apoiar a interpretação em cenários críticos, reforçando a utilidade prática do sistema sem aumentar a complexidade da decisão principal.

No conjunto, os resultados mostram que a integração de visão computacional em máquinas de autoatendimento pode conciliar eficácia na deteção de ameaças com as restrições impostas pelo hardware disponível. O sistema constitui um contributo prático, validado em condições próximas da realidade, e demonstra flexibilidade para evoluir com novas tecnologias e cenários de aplicação. Para além de reforçar os mecanismos tradicionais de segurança, acrescenta uma camada inteligente e adaptável, capaz de apoiar a resposta a situações críticas e de abrir caminho a futuros avanços na interseção entre visão computacional e sistemas de autoatendimento.

\section{Trabalho Futuro}

A experiência obtida com o desenvolvimento e a validação do sistema revelou diversas oportunidades de melhoria e de evolução que poderão ser exploradas em trabalhos futuros.

Em primeiro lugar, a atual implementação está focada na execução local. No entanto, a arquitetura do sistema já prevê a possibilidade de um modelo híbrido, no qual os modelos de inferência pré-treinados seriam executados em servidores locais com maior capacidade computacional e comunicariam com o sistema de segurança instalado na máquina de autoatendimento. Esta abordagem permitiria maior flexibilidade e escalabilidade, embora não tenha sido implementada nesta fase.

Outro ponto de evolução prende-se com a integração entre o serviço de emoções e a métrica de proximidade. Atualmente, o sistema considera no máximo duas faces para a decisão de alarme: a principal (utilizador) e a secundária. É ainda calculado o rácio entre as áreas das faces para estimar a proximidade da secundária em relação ao utilizador, embora esta métrica seja usada apenas para gerar avisos, uma evolução natural seria combinar estas duas informações, reportando emoções da face secundária apenas quando o rácio ultrapassasse o limiar definido. Esta abordagem poderia reduzir interferências de pessoas em segundo plano, mas exige uma análise cuidada para não descartar expressões relevantes, como sinais de agressividade de indivíduos mais distantes.

No que respeita à validação, embora neste trabalho tenha sido criado um \textit{dataset} de imagens (reais, simuladas e sintéticas), seria importante recolher vídeos diretamente em máquinas de autoatendimento em funcionamento. Este \textit{dataset} de validação com dados reais e contínuos permitiria aproximar a avaliação das condições efetivas de utilização, analisando a consistência temporal e a capacidade de resposta do sistema em cenários prolongados.

Outra linha de evolução relevante diz respeito à redução dos falsos positivos. Na construção inicial do sistema, a prioridade foi assegurar uma elevada sensibilidade (\textit{recall}), de forma a minimizar o risco de falhar situações de alarme. Para isso, foram aplicadas técnicas de fusão que favoreceram o \textit{recall}, o que resultou num aumento do número de alertas indevidos. Para uma utilização prática em cenários reais, torna-se agora importante reequilibrar esta estratégia, procurando reduzir falsos positivos sem comprometer a capacidade de deteção de eventos críticos.

Adicionalmente, poderá ser explorada a utilização de fusão aprendida através de regressão logística regularizada (\textit{stacking}), capaz de calibrar pesos automaticamente e reduzir redundâncias entre modelos correlacionados. Esta abordagem seria especialmente útil em configurações estáveis de clientes, onde os modelos escolhidos não sofrem alterações frequentes, permitindo criar uma fusão otimizada e especializada para esse cenário. Contudo, em contextos em que a flexibilidade de ativar ou desativar modelos é essencial, tal solução teria limitações, pois exigiria uma nova calibração sempre que a configuração fosse modificada.

Outro aspeto importante é a avaliação da credibilidade dos modelos de visão computacional integrados como serviços no sistema de segurança. A inclusão de um mecanismo interno de validação permitiria aferir não só a qualidade técnica dos modelos adicionados, mas também reduzir o risco de incorporação de componentes potencialmente maliciosos ou manipulados. Desta forma, garantir-se-ia que apenas modelos fiáveis e devidamente certificados seriam utilizados, reforçando a confiança e a robustez do sistema no seu funcionamento contínuo.

Outro caminho relevante é investir no treino de modelos mais especializados, construídos a partir de dados representativos do contexto real das máquinas de autoatendimento. Atualmente, os modelos pré-treinados utilizados foram treinados com \textit{datasets} públicos e genéricos relacionados às suas categorias. A criação de modelos especializados, por exemplo para a deteção de armas, análise emocional ou comportamentos violentos, com base em dados recolhidos em ambientes reais ou simulados, centrados em interações típicas e situações de risco concretas, permitirá construir modelos mais adaptados aos desafios do domínio. Esta abordagem não só tem potencial para aumentar a precisão da deteção, como também para reduzir significativamente a ocorrência de falsos positivos e negativos. A longo prazo, estes modelos especializados poderão substituir alguns dos modelos atualmente utilizados, contribuindo para um sistema de segurança mais fiável e sintonizado com o seu cenário de aplicação. No caso da máquina de autoatendimento HEFESTO, presente em países como Angola, torna-se importante considerar fatores culturais e demográficos no treino dos modelos, de modo a garantir maior precisão e reduzir enviesamentos. Por exemplo, no reconhecimento emocional, os dados devem incluir expressões faciais de pessoas com traços étnicos representativos da população local, enquanto na deteção de armas as imagens de treino devem refletir variações como o tom de pele das mãos que seguram os objetos, assegurando que o sistema responda adequadamente à diversidade dos diversidade do público-alvo.

Por fim, recomenda-se a avaliação do sistema de segurança em máquinas que disponham de GPU. A comparação do desempenho em cenários com e sem aceleração gráfica permitirá compreender o impacto real dessa diferença e avaliar a viabilidade de implementar o sistema em máquinas de autoatendimento com maior capacidade computacional, ponderando se os ganhos em tempo de resposta justificam o investimento.

Em conjunto, estas propostas representam caminhos viáveis e úteis para a evolução do sistema de segurança, aumentando a sua eficácia, adaptabilidade e robustez perante novos desafios.
