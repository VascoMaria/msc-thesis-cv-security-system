%\pagestyle{empty}

% ----------------------------------------------------------------------
% Página do resumo em Português:
\selectlanguage{portuguese}
\vspace*{2cm}
\begin{center} \Large \bf Resumo
\end{center}
\vspace*{1cm} \setlength{\baselineskip}{0.6cm}



As máquinas de autoatendimento fazem parte do quotidiano e assumem um papel central na prestação de diversos serviços, permitindo a realização de operações de forma rápida, autónoma e acessível. Contudo, a sua utilização crescente levanta desafios significativos no domínio da segurança, exigindo mecanismos capazes de observar, interpretar e reagir a situações de risco em tempo real.

No âmbito de um estágio, foi desenvolvido um sistema de segurança para máquinas de autoatendimento com o objetivo de reforçar a proteção dos equipamentos e assegurar que as operações decorrem de forma segura e fiável, contribuindo também para a segurança dos utilizadores. A solução recorre a técnicas de visão computacional e utiliza a câmara incorporada na máquina para analisar continuamente o ambiente e as interações que nele ocorrem, identificando sinais visuais de risco e fornecendo informação que apoia decisões rápidas e fundamentadas.

A simplicidade de integração foi um princípio orientador no desenvolvimento, permitindo que o sistema possa ser instalado sem alterar a aplicação principal da máquina, preservando todas as funcionalidades originais. Esta abordagem reduz o impacto na operação, facilita a adoção em diferentes contextos e diminui os custos de implementação.

O sistema foi concebido com uma arquitetura flexível e evolutiva, possibilitando a integração eficiente de novos modelos e tecnologias de visão computacional de forma ágil, acompanhando os avanços da área e adaptando-se a necessidades futuras.

Os resultados obtidos demonstram que a solução é viável para utilização em máquinas de autoatendimento, sendo capaz de operar de forma contínua e de sinalizar eventos críticos com latência compatível com o contexto operacional. A validação realizada com dados representativos de diferentes cenários de utilização confirma um desempenho consistente, com um valor de recall de aproximadamente 0.77, evidenciando a capacidade do sistema para identificar eficazmente situações de risco sem interferir com o funcionamento normal da máquina.

\vfill

\begin{flushleft}
\textbf{Palavras-chave:}
Máquinas de autoatendimento, Visão computacional, Deteção em Tempo Real, Análise Comportamental, Monitorização Contínua
\end{flushleft}

%\LIMPA
% Fim da página do resumo em Português
% ----------------------------------------------------------------------
