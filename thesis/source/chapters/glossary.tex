\glossary{name=CPU,description=Central Processor Unit}
\glossary{name=MMU,description=Memory Management Unit}


\newglossaryentry{Random Forest}{name=Random Forest, description={Consiste num conjunto de árvores de decisão treinadas sobre diferentes 
subconjuntos dos dados, cujas previsões são combinadas de forma agregada. }}


\newglossaryentry{VisaoComputacional}{
  name={Visão Computacional},
  description={Área da inteligência artificial que permite às máquinas interpretar e processar informação visual proveniente de imagens ou vídeos}
}

\newglossaryentry{ML}{
  name={Aprendizagem Automática (Machine Learning)},
  description={Subárea da inteligência artificial que utiliza algoritmos capazes de aprender a partir de dados, melhorando o desempenho em tarefas específicas sem programação explícita}
}


\newglossaryentry{YOLO}{
  name=YOLO,
  description={You Only Look Once, família de modelos de deteção de objetos em tempo real, reconhecida pela rapidez e precisão}
}

\newglossaryentry{NMS}{
  name={NMS (Non-Maximum Suppression)},
  description={Técnica de pós-processamento usada em modelos de deteção de objetos para eliminar previsões redundantes e manter apenas as mais relevantes}
}

\newglossaryentry{Inferencia}{
  name=Inferência,
  description={Processo de executar um modelo previamente treinado para obter previsões sobre novos dados}
}

\newglossaryentry{Batch}{
  name={Batch Processing (Processamento em Lote)},
  description={Envio de vários exemplos (imagens ou frames) em conjunto para otimizar a eficiência durante a inferência}
}

\newglossaryentry{EdgeComputing}{
  name={Edge Computing},
  description={Paradigma de computação que realiza o processamento localmente, no dispositivo ou próximo da fonte de dados, reduzindo a dependência da cloud e a latência}
}


\newglossaryentry{FastAPI}{
  name=FastAPI,
  description={Framework em Python para construção de APIs rápidas, usada no sistema de segurança para integração modular dos modelos}
}

\newglossaryentry{arrayNumPy}{
  name={Array NumPy},
  description={Estrutura de dados multidimensional da biblioteca NumPy, usada para manipulação eficiente de dados numéricos}
}

\newglossaryentry{tensor}{
  name={Tensor},
  text={tensor},
  description={Estrutura de dados utilizada em aprendizagem profunda, equivalente a um array multidimensional, fundamental em frameworks como PyTorch e TensorFlow}
}

\newglossaryentry{GPU}{
  name=GPU,
  description={Graphics Processing Unit, unidade de processamento originalmente desenvolvida para gráficos, mas amplamente utilizada para acelerar cálculos em aprendizagem profunda}
}

\newglossaryentry{datasets}{
  name={Datasets},
  text={datasets},
  description={Conjunto de dados utilizado para validar e testar modelos de aprendizagem automática}
}

\newglossaryentry{groundtruths}{
  name={Ground Truths},
  text={ground truths},
  description={Valor real ou anotação de referência usada para avaliar o desempenho de um modelo}
}

\newglossaryentry{Precision}{
  name=Precisão (Precision),
  description={Métrica que mede a proporção de previsões positivas corretas em relação ao total de previsões positivas feitas pelo modelo}
}

\newglossaryentry{Recall}{
  name=Revocação (Recall),
  description={Métrica que avalia a proporção de previsões positivas corretas em relação ao total de casos positivos existentes no conjunto de dados}
}

\newglossaryentry{F1}{
  name=F1-score,
  description={Métrica que combina precisão e revocação numa média harmónica, equilibrando ambas}
}

\newglossaryentry{estimativaPose}{
  name={Estimativa de Pose},
  text={estimativa de pose},
  description={Técnica de visão computacional que identifica e rastreia a posição das articulações do corpo humano, permitindo analisar movimentos e comportamentos}
}

\newglossaryentry{vgg19}{
  name={VGG-19},
  description={Arquitetura de rede neuronal convolucional profunda com 19 camadas, desenvolvida pelo \textit{Visual Geometry Group} da Universidade de Oxford}
}
\newglossaryentry{opencv}{
  name={OpenCV},
  description={Biblioteca de código aberto de visão computacional (\textit{Open Source Computer Vision Library}). Fornece ferramentas para processamento de imagem e vídeo, deteção e descrição de características, calibração de câmara, rastreamento, e integrações com aprendizagem automática. Suporta C/C++, Python e outras linguagens}
}

\newglossaryentry{restful}{
  name={RESTful},
  description={Estilo arquitetural para serviços Web baseado nos princípios do \gls{rest} (\textit{Representational State Transfer}). Um sistema RESTful segue estas restrições de forma consistente, usando operações HTTP para criar, ler, atualizar e remover recursos identificados por URLs}
}

\newglossaryentry{fastapi}{
  name={FastAPI},
  description={\textit{Framework} moderna em Python para criação de APIs rápidas e seguras, baseada nos princípios RESTful. É otimizada para alto desempenho utilizando \textit{Starlette} e \textit{Pydantic}, com suporte a tipagem estática, documentação automática (OpenAPI/Swagger) e integração fácil com aplicações de computação visual, inteligência artificial e micro-serviços}
}

\newglossaryentry{cache}{
  name={Cache},
  text={cache},
  description={Mecanismo de armazenamento temporário que guarda dados ou resultados frequentemente utilizados, de modo a reduzir o tempo de acesso e melhorar o desempenho do sistema. Pode existir em vários níveis, como memória, disco ou aplicações}
}

\newglossaryentry{python}{
  name={Python},
  description={Linguagem de programação de alto nível, interpretada e de propósito geral. É conhecida pela sua sintaxe simples e legibilidade, possuindo uma vasta biblioteca padrão e ecossistema de pacotes. É amplamente utilizada em áreas como ciência de dados, inteligência artificial, computação visual e desenvolvimento de aplicações Web}
}

\newglossaryentry{bytes}{
  name={Bytes},
  description={Unidade de informação digital composta, em geral, por 8 bits. É utilizada para representar e armazenar dados em sistemas informáticos, como caracteres de texto, instruções de máquina ou elementos multimédia. No plural, \textit{bytes} refere-se a múltiplas unidades desta medida}
}
\newglossaryentry{fer2013}{
  name={FER-2013},
  description={Conjunto de dados (\textit{dataset}) para reconhecimento de expressões faciais, criado em 2013. É amplamente utilizado em tarefas de treino e avaliação de modelos de \gls{fer} (\textit{Facial Emotion Recognition})}
}

\newglossaryentry{rafdb}{
  name={RAF-DB},
  description={\textit{Real-world Affective Faces Database}. Conjunto de dados de faces em contexto real, anotadas com diferentes expressões emocionais. É amplamente utilizado em tarefas de reconhecimento de emoções faciais}
}

\newglossaryentry{rwf2000}{
  name={RWF-2000},
  description={\textit{Real-world Fighting Dataset 2000}. Conjunto de dados composto por 2000 vídeos recolhidos de ambientes reais, utilizados para treinar e avaliar modelos de deteção de comportamentos violentos}
}
